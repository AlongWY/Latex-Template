\documentclass{BaseSetting} 

\def\publishdate{2018年9月}

\def\clubname{IBM俱乐部}

\pagestyle{empty}
\begin{document}

\begin{center}
    \Large \textbf{哈尔滨工业大学学生社团指导教师确认函}
\end{center}

哈尔滨工业大学学生社团发展至今已经有了相当的规模,在丰富同学课余生活,全面提高同学素质,充实我校的第二课堂等方面都起着很重要的作用。

指导教师是社团工作的重要指导者,对社团的发展有着非常重要的管理职能。为了更好的促进社团的发展,方便您对社团工作的指导,我们将指导教师所需要参与的部分工作大致罗列如下:

\begin{enumerate}
\item 社团章程需要老师审阅。
\item 学期初的注册。社团每学期注册一次,学生社团联合会将与您沟通确认。
\item 社团举办的涉外或者大型活动需要指导教师的审查批准。社团举办的大型及涉外活动在校内外都有较强的影响力,指导教师的审查既可以使活动顺利开展,又可以避免活动过程中出现不必要的问题。
\item 社团的换届需要指导教师的指导。会长的更换、社团领导班子的更替需要在一定的指导下完成,才能保证社团发展的连续性。
\item 社团的日常活动需要指导教师的参与。经常参与社团活动,可以使指导教师更加了解所指导的社团,可以与社团成员之间有更多的接触,加深师生间的友谊。
\item 社团活动的资源需要指导教师的支持。这里所指的资源包括活动场地、信息沟通或者是资金来源等等。校内的活动资源相对紧缺,社团活动的举办也存在着资金与场地的压力,同学们所能够掌握的信息有限,这些都大大限制了社团的快速发展。因此指导教师的帮助将是社团发展最直接的推动力!
\end{enumerate}

感谢您对学生社团工作的支持,您的指导将对社团的工作产生意义深远的影响。我们谨代表各类社团和广大参与社团活动的同学向您表示由衷的感谢!

\begin{flushright}
共青团哈尔滨工业大学委员会

哈尔滨工业大学学生社团联合会

\publishdate

\end{flushright}

\begin{center}
\rule{\linewidth}{0.2mm}

\textbf{确认函}
\end{center}

我已经阅读并理解了《哈尔滨工业大学学生社团指导教师确认函》的内容,同意哈尔滨工业大学 \clubname 的《章程》,并承诺对该社团的工作给予必要的指导与管理。 


\begin{flushright}
签字:\hspace{0.2\linewidth}

年\hspace{0.05\linewidth}月\hspace{0.05\linewidth}日
\end{flushright}

\end{document}